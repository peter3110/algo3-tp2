% ------ headers globales -------------
\documentclass[11pt, a4paper, twoside]{article}
\usepackage{header}
\usepackage{config}
% -------------------------------------
\begin{document}

% -- Carátula --
%\clearpage{\pagestyle{empty}\input{caratula}\cleardoublepage}

%% =====================================================================

\clearpage
\section{Problema 1 - Plan de vuelo}

\begin{figure}[H]
\centering
\includegraphics[scale=0.5]{imagenes/plane.jpg}
\end{figure}

\subsection{Descripción del problema a resolver}
En el presente problema, nos encontramos con una lista de vuelos internacionales
entre distintos países. Cada vuelo tiene un origen, un destino y horarios de 
partida y llegada a cada uno de estos. Además, nos dan una ciudad A desde
la cuál hay que partir y otra B a la cuál deseamos llegar. El objetivo es determinar un 
itinerario (es decir una combinación de los vuelos disponibles) que permita llegar 
de A a B lo más temprano posible, sin importar la cantidad de vuelos utilizada. Para
hacer combinación entre dos vuelos, tienen que haber dos horas de diferencia entre los
horarios de llegada y partida de ambos del aeropuerto. Veamos un ejemplo de instancia 
para el problema: \\

\begin{minipage}[t]{0.4\textwidth}
\begin{Verbatim}[frame=single,framesep=1cm,label= Ejemplo de instancia 1]
bragado ezeiza 5
bragado rosario 3 5
rosario catamarca 7 9
bragado ezeiza 1 18
catamarca ezeiza 11 15
rosario ezeiza 7 100
\end{Verbatim}
\end{minipage}
\hfill
\begin{minipage}[t]{0.4\textwidth}
\begin{Verbatim}[frame=single,framesep=1cm,label= Ejemplo de instancia 2]
bragado ezeiza 5
bragado rosario 3 5
rosario catamarca 7 9
catamarca bragado 0 1
catamarca ezeiza 10 15
rosario ezeiza 7 12
\end{Verbatim}
\end{minipage}

\begin{minipage}[t]{0.4\textwidth}
\begin{Verbatim}[frame=single,framesep=1cm,label= Salida para instancia 1]
15 3 1 2 4
\end{Verbatim}
\end{minipage}
\hfill
\begin{minipage}[t]{0.4\textwidth}
\begin{Verbatim}[frame=single,framesep=1cm,label= Salida para instancia 2]
12 2 1 5
\end{Verbatim}
\end{minipage}

En los ejemplos 1 y 2 se quiere encontrar un itinerario para viajar de bragado a 
ezeiza, elijiendo algunos de los 5 vuelos disponibles. En el primer ejemplo, se 
puede llegar como mínimo en el horario t = 15, tomando 3 vuelos. Y uno de los
posibles itinerarios (podría haber más de uno que use 3 vuelos) sería tomar los 
vuelos 1, 2 y 4, en ese orden. Este itinerario de vuelos no podría realizarse en
el ejemplo 2, porque el vuelo que une catamarca-ezeiza sale en t = 10, pero el 
único vuelo que llega a catamarca, lo hace en t = 9. Como hay menos de dos horas
de diferencia entre que llega un vuelo y sale el otro, no es posible hacer esta 
combinación. Además, en el segundo ejemplo el mejor horario en el que se puede 
llegar a ezeiza es en t = 12, tomando los vuelos 1 y 5. \\

\subsection{Ideas desarrolladas para la resolución}
Para resolver este problema, consideramos a cada vuelo como un nodo. En base a 
las ciudades de origen y destino y horarios de salida y llegada de cada uno de
los vuelos, utilizamos la técnica de programación dinámica para asignar a cada
par de nodos una arista que los una si es que hay algún itinerario posible que 
permita pasar por ambos vuelos. Así, generamos un grafo dirigido en el cual para 
todo par de nodos A y B, hay una arista que los une si y sólo si se puede llegar 
desde el nodo A al nodo B, pasando por algún subconjunto del resto de los nodos.
Representamos este grafo dirigido como una matriz A de $nxn$ donde n es la cantidad
de nodos y donde $A_{ij}$ es 1 si hay un camino que une al nodo i con el nodo j y
0 si no. Esta matriz es llamada comúnmente la matriz de alcance de un digrafo. \\
Entonces, primero veamos qué es lo que hace nuestro algoritmo en los primeros 4
pasos, basándonos en el ejemplo 1: \\

\[ \left| \begin{array}{ccccc}
  0 & 1 & 0 & 0 & 1 \\
  0 & 0 & 0 & 1 & 0 \\
  0 & 0 & 0 & 0 & 0 \\
  0 & 0 & 0 & 0 & 0 \\
  0 & 0 & 0 & 0 & 0 \end{array} \right| ->
  %
  \left| \begin{array}{ccccc}
  0 & 1 & 0 & 1 & 1 \\
  0 & 0 & 0 & 1 & 0 \\
  0 & 0 & 0 & 0 & 0 \\
  0 & 0 & 0 & 0 & 0 \\
  0 & 0 & 0 & 0 & 0 \end{array} \right| ->
  %
  \left| \begin{array}{ccccc}
  0 & 1 & 0 & 1 & 1 \\
  0 & 0 & 0 & 1 & 0 \\
  0 & 0 & 0 & 0 & 0 \\
  0 & 0 & 0 & 0 & 0 \\
  0 & 0 & 0 & 0 & 0 \end{array} \right| ->
  %
  \left| \begin{array}{ccccc}
  0 & 1 & 0 & 1 & 1 \\
  0 & 0 & 0 & 1 & 0 \\
  0 & 0 & 0 & 0 & 0 \\
  0 & 0 & 0 & 0 & 0 \\
  0 & 0 & 0 & 0 & 0 \end{array} \right| ->
  %
  \left| \begin{array}{ccccc}
  0 & 1 & 0 & 1 & 1 \\
  0 & 0 & 0 & 1 & 0 \\
  0 & 0 & 0 & 0 & 0 \\
  0 & 0 & 0 & 0 & 0 \\
  0 & 0 & 0 & 0 & 0 \end{array} \right|
\] 

Se puede ver que en la primera matriz, hay un 1 en la posición $(i,j)$ sii se
puede hacer conexión directa entre los vuelos $i$ y $j$. En el siguiente paso, 
se actualiza la primera fila: desde el vuelo 1 puedo llegar a los vuelos que puedo
llegar combinando de forma directa ó a los vuelos que puedo llegar combinando 
pasando antes por un vuelo intermedio. En el ejemplo, paso del vuelo 1 al vuelo 2 
y luego, del vuelo 2 puedo pasar al vuelo 4. En los pasos siguientes la matriz
permanece inalterada porque los vuelos 3, 4 y 5 van directo a ezeiza, y no hay
ningún vuelo que salga de allí. Como ya analizamos para todo par de vuelos si era
posible llegar de uno al otro, miramos la matriz que nos quedó. Mirándola con
atención, se puede calcular, para cualquier par de ciudades, si es posible llegar
de una a la otra y, si lo es, en que horario. \\
Veamos ahora con el ejemplo 2: \\

\[ \left| \begin{array}{ccccc}
  0 & 1 & 0 & 0 & 1 \\
  0 & 0 & 0 & 0 & 0 \\
  1 & 0 & 0 & 0 & 0 \\
  0 & 0 & 0 & 0 & 0 \\
  0 & 0 & 0 & 0 & 0 \end{array} \right| ->
  %
  \left| \begin{array}{ccccc}
  0 & 1 & 0 & 0 & 1 \\
  0 & 0 & 0 & 0 & 0 \\
  1 & 0 & 0 & 0 & 0 \\
  0 & 0 & 0 & 0 & 0 \\
  0 & 0 & 0 & 0 & 0 \end{array} \right| ->
  %
  \left| \begin{array}{ccccc}
  0 & 1 & 0 & 0 & 1 \\
  0 & 0 & 0 & 0 & 0 \\
  1 & 1 & 0 & 0 & 1 \\
  0 & 0 & 0 & 0 & 0 \\
  0 & 0 & 0 & 0 & 0 \end{array} \right| ->
  %
  \left| \begin{array}{ccccc}
  0 & 1 & 0 & 0 & 1 \\
  0 & 0 & 0 & 0 & 0 \\
  1 & 1 & 0 & 0 & 1 \\
  0 & 0 & 0 & 0 & 0 \\
  0 & 0 & 0 & 0 & 0 \end{array} \right| ->
  %
  \left| \begin{array}{ccccc}
  0 & 1 & 0 & 0 & 1 \\
  0 & 0 & 0 & 0 & 0 \\
  1 & 1 & 0 & 0 & 1 \\
  0 & 0 & 0 & 0 & 0 \\
  0 & 0 & 0 & 0 & 0 \end{array} \right|
\] 

En este ejemplo, a diferencia de lo que sucede en el primero, en la posición 
$(2,4)$ no hay un 1, pues no se puede combinar de forma directa a los vuelos
rosario-catamarca con catamarca-ezeiza, por un tema de horarios de salida-llegada.
En este caso, la posición $(1,3)$ es igual a 1, porque se puede conectar de forma
directa a los vuelos catamarca-bragado y bragado-rosario. En el primer paso no se
puede agregar ninguna combinación para el vuelo 1, porque los vuelos 2 y 5 llegan
ambos a ezeiza, y no hay ningún vuelo que salga de allí. En el siguiente paso, como
el vuelo 2 llega a ezeiza, no se puede hacer nada por la misma razón. En el tercer 
paso, se puede añadir las combinaciones entre los vuelos 3-1 y 3-5, pues el vuelo 3
llega a bragado y de bragado salen dos vuelos: el 1 y el 5. Por la misma razón
que no se pudo hacer nada para el vuelo 2, tampoco se puede hacer nada para los vuelos
4 y 5 y en los dos últimos pasos. \\

Entonces, como se ve en los ejemplos de más arriba, lo que hace nuestro algoritmo
es ir vuelo por vuelo, actualizando para cada uno los vuelos a los cuáles puede
ir y desde los cuáles se puede llegar. Al terminar de actualizar el último vuelo,
en la fila $i$ de nuestra matriz va a formar un vector $v_{i}$, que en la posición
$j$ va a tener un 1 si y sólo si hay un posible itinerario de vuelos que permite
llegar desde el vuelo $i$ al vuelo $j$. \\

\subsection{Justificación de cota de complejidad}
Para poder cumplir con la complejidad requerida, que es de $O(n^2)$, representamos
la matriz de la cuál hablamos en la sección anterior como un vector de enteros.
De esta forma..

\subsection{Testeos de complejidad}



%% =====================================================================

\end{document}
