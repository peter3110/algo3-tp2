% ------ headers globales -------------
\documentclass[11pt, a4paper, twoside]{article}
\usepackage{header}
\usepackage{config}
\usepackage{graphicx}
\usepackage{amsthm}
\usepackage[lined, boxed, commentsnumbered]{algorithm2e}
% -------------------------------------
\begin{document}

% -- Carátula --
%\clearpage{\pagestyle{empty}\input{caratula}\cleardoublepage}

%% =====================================================================

\clearpage
\section{Problema 3 - La comunidad del anillo}

\subsection{Descripción del problema a resolver}
La empresa \texttt{AlgoNET} nos contrató para que le brindemos una 
solución algorítmica a un problema de redes. Nuestro cliente quiere ofrecer
un sevicio particular sobre una red existente de computadoras. La red cuenta
con un conjunto de conexiones entre pares de computadoras, donde cada 
conexión tiene un costo asociado. El objetivo es elegir un subconjunto de 
conexiones y de equipos (que funcionarán como servidores) tales que:
\begin{enumerate}
  \item El conjunto de servidores deberá conformar un anillo
  \item Todo equipo debe quedar conectado al anillo de servidores vía un 
        enlace directo o pasando a través de otros equipos en el medio
  \item La suma total de los costos de todos los enlaces utilizados debe
        ser lomínimo posible
  \item El algoritmo debe tener una complejidad estrictamente mejor que $O(n^3)$
  \item El algoritmo debe detectar los casos en los que no haya solución
\end{enumerate}

El formato de entrada contiene una instancia del problema. La primera línea 
contiene un entero positivo $n$ que indica la cantidad de equipos de la red
(numerados de 1 a $n$), y un entero no negativo $m$ que corresponde a la 
cantidad de enlaces disponibles. A esta línea le siguen $m$ líneas, una para
cada enlace, con el formato \texttt{e1 e2 c} donde \texttt{e1} y \texttt{e2}
representan los equipos en lso extremos del enlace en cuestión (ambos enteros
entre 1 y $n$) y \texttt{c} es el costo por utilizar dicho enlace. En caso de
haber solución, la salida tiene el formato \texttt{C Ea Er} donde \texttt{C} es
el csoto de la solución dada y \texttt{Ea} y \texttt{Er} son los enlaces 
utilizados por el anillo y para el resto de la red, respectivamente. A esta línea
le siguen \texttt{Ea} líneas, una para cada enlace utilizado en el anillo y luego
\texttt{Er} líneas, una para cada enlace utilizado fuera del anillo, todas con el 
formato \texttt{e1 e2} donde \texttt{e1} y \texttt{e2} representan los extremos
del enlace en cuestión. Si hay más de una solución óptima, se puede devolver 
cualquier, y si no hay solución se debe devolver la palabra \texttt{no}. A 
continuación, mostramos un ejemplo de una posible instancia.

\begin{minipage}[t]{0.4\textwidth}
\begin{Verbatim}[frame=single,framesep=1cm,label= Ejemplo de entrada: instancia 1]
8 10
1 2 1
2 6 1
2 3 5
2 7 2
6 7 1
7 5 3
3 5 8
5 4 2
5 8 2
4 8 10
\end{Verbatim}
\end{minipage}
\hfill
\begin{minipage}[t]{0.4\textwidth}
\begin{Verbatim}[frame=single,framesep=1cm,label= Ejemplo de salida: instancia 1]
20 3 5
2 6
6 7
2 7
1 2
7 5
3 5
5 4
5 8
\end{Verbatim}
\end{minipage}

A continuación, presentamos una representación del problema, visto como 
un grafo: cada nodo es un equipo y cada arista es una conexión entre 
algún par de computadoras, y tiene un costo asociado a dicha conexión, 
seguido de tres posibles formas de elegir los subconjuntos de equipos y 
conexiones requeridos por el problema

\begin{figure}[H]
\includegraphics[scale=1]{imagenes/equipos2.png}
\caption{Red de equipos}
\includegraphics[scale=1]{imagenes/equipos3.png}
\includegraphics[scale=1]{imagenes/equipos4.png}
\includegraphics[scale=1]{imagenes/equipos5.png}
\caption{Soluciones de costo C = 20 (rojo), C = 24 (azul), C = 25 (marrón) respectivamente}
\end{figure}

Se puede observar que las tres soluciones presentadas en la figura 1.1.2 
son efectivamente soluciones porque todas presentan algún circuito de
servidores, y en todos los casos, todas las computadoras de la red están 
conectadas con los servidores con conexiones directas o a través de conexiones
con otros equipos. Sin embargo, no todas estas soluciones son óptimas: una 
tiene costo 20, otra 24 y otra 25. Es decir, que para este caso en particular,
nuestro algoritmo debería devolver como solución los servidores 2, 6, 7 y
los enlaces que están marcados en rojo en la figura 1.1.2.

\subsection{Ideas desarrolladas para la resolución}
A partir del ejemplo presentado en la sección anterior y de algunos otros,
observamos que en todos los casos, todos los enlaces que conformaban el 
árbol generador mínimo (AGM) del grafo generados por nuestra red de equipos
estaban incluidos en la solución óptima del problema. En nuestro ejemplo,
el AGM sería:

\begin{figure}[H]
\centering
\includegraphics[scale=1]{imagenes/equipos6.png}
\caption{Árbol Generador Mínimo del grafo del ejemplo de la figura 1.1.1}
\end{figure} 

Entonces, pensamos que posiblemente esto siempre sucediera, al menos siempre
que existiera alguna solución. Así es que,
intentamos probar que tomando el AGM del grafo en cuestión y luego 
agregándole la arista de menor pero de las que no estuvieran incluidas
en el AGM, el grafo generado sería una solución (sería un grafo con algún 
ciclo y todos los nodos estarían conectados con algún nodo del conjunto de 
nodos que conformaban el ciclo en el grafo) y que sería óptima (la suma
total de los pesos de los enlaces sería lo mínimo posible). La demostración
está más adelante. \\
Entonces, a continuación describimos los pasos que nuestro algoritmo
efectuará para resolver el problema planteado:

\begin{enumerate}
  \item generar un grafo $G$ en donde cada nodo sea un equipo y cada arista un
        enlace entre un par de equipos, con un costo asociado igual al costo
        que hay que pagar por el ancho de banda empleado por dicha conexión
  \item chequear si el grafo $G$ es conexo. Si no lo es, entonces el problema
        no tendrá solución
  \item si el grafo $G$ es conexo, generar el árbol generador mínimo $T$ del grafo
        $G$
  \item buscar el menor de los enlaces que aparece en el grafo $G$ pero que no 
        aparece en el árbol generador mínimo $T$. Llamemos a ese eje $e$.
  \item agregamos el eje $e$ al AGM $T$. Queda así formado el grafo $G' = T + e$,
        que es un grafo conexo y tiene un único ciclo. Además, cumple con todo lo
        que nos pide el enunciado del problema
  \item devolvemos la suma total de los costos de todas las conexiones que son
        utilizadas por nuestro grafo $G'$ y, por otro lado, devolvemos los enlaces
        que conforman nuestro anillo de servidores y el resto de los enlaces 
        de nuestra solución
        
\end{enumerate}

\newpage
\subsection{Demostración de Correctitud}
Para demostrar la correctitud de nuestro algoritmo, tenemos que demostrar que los
pasos descriptos en la sección anterior para la resolución del problema 
efectivamente nos conducen a la generación de una solución correcta. Entonces,
repasando las ideas ya expuestas, lo que planteamos es que se puede llegar a una
solución óptima a partir de generar el AGM $T$ de nuestro grafo $G$ y luego generar
el grafo $G' = T + \{e\}$, donde $e$ es el eje de menor costo que esta en $G$ y
no en $T$. Pero lo que nos pide el enunciado es que generemos, a partir del grafo $G$,
otro grafo $G''$ tal que $G''$ sea conexo, contenga un único ciclo y que la suma
total de los costos de sus ejes sea lo mínima posible. Entonces, demostrando la 
siguiente proposición, estaríamos demostrando nuestro método empleado para la resolución
es efectivamente correcto:

\newtheorem{prop}{Proposición}
\begin{prop}
Dado un grafo G conexo, con T un AGM de G y S el conjunto de todos los árboles
generadores de G y $e$ el mínimo eje perteneciente a $G \backslash T$ vale que: 
$costo(T+e) = \min\limits_{\substack{f \in G \backslash T \\ T' \in S}} costo(T'+f)$
\end{prop}
\begin{proof}[Demostración de la Proposición 1] 
Demostraremos ambas desigualdades:
\begin{flushleft} $\leq$ : en este caso, vale que $ costo(T) \leq costo(T')$ y además $costo(e) <= costo(f) $. 
Entonces, es evidente que $ costo(T+e) = costo(T)+costo(e) \leq costo(T')+costo(f) = costo(T'+f) $  \\
$\geq$ : Tomando $T'=T$ y $f=e$, tenemos que $costo(T'+f)=costo(T+e)$. 
Entonces, sigue que $ \min\limits_{\substack{f \in G \backslash T \\ T' \in S}} costo(T'+f) \leq costo(T+e) $
\end{flushleft}
\end{proof}

\subsection{Justificación de cota de complejidad}

\begin{algorithm}[H]
\SetKwInOut{Input}{input}
\SetKwInOut{Output}{output}
  \Input{???}
  \Output{??}

\caption{Pseudocódigo para el algoritmo empleado en la resolución}
\end{algorithm}

\subsection{Testeos de complejidad}



%% =====================================================================

\end{document}
